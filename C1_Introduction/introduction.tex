\chapter{Introduction}
\label{sec:introduction}

Micro-hypervisors have been extensively used in recent research for security purposes~\cite{mccune2010trustvisor, vasudevan2012lockdown, steinberg2010nova, zhou2014dancing, zhou2012building, yu2015trusted, cui2018securing}. Exemplary security properties include memory separation~\cite{mccune2010trustvisor}, execution environment separation~\cite{vasudevan2012lockdown, steinberg2010nova}, and I/O separation~\cite{zhou2012building, zhou2014dancing, yu2015trusted, cui2018securing}. To minimize the engineering effort of building micro-hypervisors from scratch, XMHF~\cite{vasudevan2013design} implements an extensible micro-hypervisor framework that encapsulates core functionalities of general micro-hypervisors and supports running a single rich guest operating system (OS). Other projects can extend XMHF with hypapps that implement custom security properties. The XMHF core consists of only around 6000 lines of code. The XMHF core is formally verified using the CBMC model checker to ensure that it protects the memory integrity of the micro-hypervisor (i.e., XMHF and its hypapp).

\section{Problems}

XMHF does not meet the requirements of modern x86 hardware and OSes in several ways. XMHF only supports 32-bit rich OSes, but not 64-bit ones. Additionally, XMHF does not support TPM 2.0, which is commonly found on new hardware. As a result, XMHF does not support modern operating systems such as Windows 10 and Debian 11.

XMHF does not support virtualization of the hardware virtualization extension, so rich OSes cannot utilize this feature of modern CPUs. As a result, rich applications that rely on general hypervisors like VirtualBox and Hyper-V cannot run on top of XMHF.

\section{Challenges}

In this thesis, we present \XMHF64, which enhances XMHF by supporting modern OSes and hardware.

Improving XMHF to support new commodity OSes on new hardware is non-trivial. First, the micro-architectural requirements of modern OSes must be identified and fulfilled by \XMHF64. For example, \XMHF64 must virtualize the hardware virtualization extension in order to support modern general-purpose hypervisors.

Second, \XMHF64 must maintain the security properties of XMHF. Although formal verification of \XMHF64 is left as future work, \XMHF64 must be designed to be verifiable. Therefore, \XMHF64 must follow XMHF's design to minimize additional verification efforts while adding the necessary functionalities to support new OSes and hardware.

\section{Contributions}

We demonstrate that micro-hypervisors can be enhanced to support the latest OSes and hardware, while still isolating security-sensitive programs from untrusted OSes and applications. We present the design and implementation of \XMHF64, which improves upon XMHF by adding support for modern 64-bit operating systems and Intel hardware. We implement virtualization of the hardware virtualization extension in \XMHF64 to support general-purpose hypervisors. We show that the memory integrity of \XMHF64 and its hypapp is verifiable, though its formal verification is left as future work. TrustVisor \cite{mccune2010trustvisor} is a micro-hypervisor and is ported to run on top of XMHF \cite{vasudevan2013design}. In this thesis, we find and fix six security bugs in TrustVisor when running as a hypapp in XMHF.

The rest of this thesis is structured as follows.
Section~\ref{sec:background} discusses background information related to micro-hypervisors.
Section~\ref{sec:design_implementation} describes the design and implementation of \XMHF64.
Section~\ref{sec:evaluation} evaluates the new \XMHF64 implementation based on code size, compatibility, and performance.
Section~\ref{sec:vulnerabilities_xmhf_tv} discusses the security vulnerabilities found in TrustVisor.
Section~\ref{sec:discussion_future_work} describes findings during \XMHF64's development and future work.
Section~\ref{sec:conclusion} concludes the thesis.

