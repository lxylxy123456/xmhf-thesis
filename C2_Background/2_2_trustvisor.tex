\subsection{TrustVisor}
\label{sec:bg_tv}

TrustVisor \cite{mccune2010trustvisor} is a micro-hypervisor that supports running security-sensitive programs that are separated from the rich OS/Apps. In TrustVisor, the security-sensitive programs are called pieces of application logic (PALs). TrustVisor protects both the secrecy and integrity of PALs from the rich OS/Apps. \cite{vasudevan2013design} ports TrustVisor to run as a hypapp in XMHF.

The life cycle of a PAL consists of four types of events, as discussed below.

\begin{itemize}
\item \textbf{PAL Registration}: The rich OS/Apps register a PAL to TrustVisor, specifying the memory regions in which the PAL operates, the argument format for calling the PAL, and the entry point of the PAL. TrustVisor moves the PAL's memory from the XMHF primary partition to a newly-created XMHF secondary partition. This ensures memory separation between the rich OS/Apps and the PAL.

\item \textbf{PAL Invocation}: The rich OS/Apps call the entry point of the PAL using a C function call. TrustVisor copies all arguments from the rich OS/Apps to the PAL. Then, TrustVisor executes the PAL. While the PAL is executing, interrupts are disabled to ensure that the PAL executes in a single-threaded environment.

\item \textbf{PAL Termination}: The PAL completes its execution and executes a C function return. TrustVisor copies the return value from the PAL to the rich OS/Apps. Then, TrustVisor resumes the rich OS/Apps.

\item \textbf{PAL Unregistration}: The rich OS/Apps unregister a PAL from TrustVisor. TrustVisor clears the PAL's memory, which may contain secret information, and then moves the PAL's memory from the XMHF secondary partition back to the XMHF primary partition.
\end{itemize}

