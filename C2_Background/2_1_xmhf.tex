\subsection{XMHF}
\label{sec:bg_xmhf}

XMHF \cite{vasudevan2013design} is a framework that assists the development of micro-hypervisors. The XMHF core implements the common logic in micro-hypervisors that interacts with the hardware virtualization extension. Developers can extend the functionality of XMHF through hypapps. Thus, different hypapps can share the same XMHF core, reducing development cost.

\begin{figure}[tbp]
	\begin{center}
	\includestandalone[width=\textwidth]{xmhf_arch}
	\end{center}
	\caption[XMHF architecture.]{XMHF architecture, adapted from \cite{vasudevan2013design}.}
	\label{fig:xmhf_arch}
\end{figure}

The architecture of XMHF is shown in \ref{fig:xmhf_arch}. XMHF supports only one full-featured rich OS and its applications running in the primary partition. The hypapps can create secondary partitions to run isolated code. XMHF provides APIs for hypapps to enforce memory separation between the primary partition and the secondary partitions.

The XMHF core has six components. The \textit{startup} component performs initialization after XMHF boots. The \textit{smpguest} component supports running XMHF on multiple CPUs. The \textit{partition} component initializes hardware virtualization extension data structures for XMHF partitions. The \textit{eventhub} component handles all intercepts from XMHF partitions. The \textit{memprot} component protects the memory integrity of XMHF and the hypapp from XMHF partitions using HPT. The \textit{dmaprot} component protects the memory integrity of the XMHF and the hypapp from I/O devices through IOMMU page tables.

All intercepts generated by XMHF partitions are handled by the \textit{eventhub} component. For intercepts related to the hypapp, as discussed below, the \textit{eventhub} component calls the corresponding callback function in the hypapp to let the hypapp handle the intercept. The intercepts related to the hypapp are:
\begin{itemize}
\item The rich OS/Apps or isolated code accesses an I/O port reserved by the hypapp using the input from port (IN) or output to port (OUT) instruction.
\item The rich OS/Apps or isolated code accesses invalid memory in the partition's HPT (i.e., HPT violation).
\item The rich OS/Apps or isolated code invokes the hypapp using the call to VM monitor (VMCALL) instruction.
\item The rich OS/Apps or isolated code tries to restart the machine.
\end{itemize}

Using the \textsc{DRIVE} methodology, \cite{vasudevan2013design} formally verifies that the XMHF core protects the memory integrity property of XMHF and the hypapp. A micro-hypervisor's memory integrity property can be formally-verified if the micro-hypervisor  follows all the six properties required by the \textsc{DRIVE} methodology as discussed in Section~\ref{sec:bg_drive}.

