\section{Future Work}

\subsection{Compatibility}

In the future, we plan to support IOMMU virtualization and UEFI booting in \XMHF64. Currently, \XMHF64 hides the IOMMU from the rich OS/Apps because general-purpose hypervisors do not require it. However, virtualization-based security (VBS) in Windows 10 requires the IOMMU \cite{windows_vbs}. UEFI booting is also required by VBS and the latest operating systems like Windows 11. After supporting IOMMU virtualization and UEFI booting, we will work on enabling VBS in Windows 10.

Though XMHF supports both Intel and AMD CPUs, \XMHF64 currently only supports Intel CPUs. We plan to add support to \XMHF64 for AMD CPUs in the future. The hardware virtualization extension in AMD CPUs is provided through Secure Virtual Machine (SVM), a technology different from VMX. Thus, \XMHF64 must also virtualize SVM to support AMD CPUs.

\subsection{Formal Verification}

The formal verification of \XMHF64's and its hypapp's memory integrity is left as future work. However, as shown in Section~\ref{sec:design_verifiability}, \XMHF64 satisfies all properties required by the \textsc{DRIVE} methodology. \cite{vasudevan2013design} formally proves the memory integrity of XMHF and its hypapp using CBMC, and we believe that a similar approach can be followed to prove the memory integrity of \XMHF64 and its hypapp.

While formally proving the memory integrity of \XMHF64 and its hypapp is important, memory integrity alone cannot guarantee the security of \XMHF64. Similar to XMHF, \XMHF64 also needs to maintain its control flow integrity (CFI) to ensure its security \cite{vasudevan2013design}. Therefore, we expect future work to address other security properties of \XMHF64, including CFI.

\subsection{Hypapps Support}

Currently, \XMHF64 only supports TrustVisor, but supporting other hypapps is left as future work. To enable a general hypapp to interact correctly with guest VMs, it may be necessary to modify the interface between \XMHF64 and hypapps. For example, \XMHF64 may need to define new callback functions.

We also find that some hypapps may not function correctly under XMHF and \XMHF64's quiescing design. For example, if a hypapp's callback function requires interaction with another CPU, the hypapp cannot run correctly because all other CPUs are quiesced. In the future, we plan to investigate how \XMHF64 can support this type of hypapp while maintaining its atomicity.

\subsection{Performance Optimization}

In the future, we plan to further increase the performance of \XMHF64 by optimizing the use of EPTs. Currently, \XMHF64 manages all rich OS/Apps memory with 4 KiB pages in EPT, resulting in significant memory overhead. For a machine with 8 GiB of memory and eight CPUs, \XMHF64 requires more than 128 MiB to store EPT. However, if \XMHF64 uses 2 MiB or 1 GiB pages in EPT, the memory overhead would be less than 0.5 MiB. Using large pages also improves performance since EPT uses fewer TLB entries.

