\chapter{Conclusion}
\label{sec:conclusion}

Micro-hypervisors are used in many research projects to improve the security of computer systems. For example, micro-hypervisors can separate security-sensitive programs from a commodity operating system, which typically contains millions of lines of code. XMHF \cite{vasudevan2013design} is a framework that facilitates the development of micro-hypervisors. However, the hardware and operating systems XMHF supports are obsolete.

In this thesis, we demonstrate that micro-hypervisors can support modern hardware and operating systems. We present \XMHF64, an enhanced version of XMHF. \XMHF64 satisfies the micro-architectural requirements of modern rich OS/Apps (e.g., Windows 10, Debian 11) by authorizing their accesses to hardware resources with three policies: exporting, hiding, and virtualization. XMHF+ also virtualizes the hardware virtualization extension to support rich OS/Apps to run general-purpose hypervisors. Moreover, XMHF+ supports 64-bit micro-architectures and TPM 2.0. 

Our compatibility improvements to \XMHF64 do not compromise its design for verifiability. \XMHF64 preserves all security properties required by the \textsc{DRIVE} methodology, which is used to prove the memory integrity of XMHF and its hypapp \cite{vasudevan2013design}. Thus, the memory integrity of \XMHF64 and its hypapp can be formally verified in future work.

In this thesis, we identify six vulnerabilities in TrustVisor when running as a hypapp in XMHF. We address these vulnerabilities in \XMHF64 and backport the fixes to XMHF. Additionally, we report multiple bugs in other software used during our development, such as compilers and general-purpose hypervisors.

